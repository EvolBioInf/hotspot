\documentclass{article}
\usepackage{graphics,eurosym,latexsym}
\usepackage{listings}
\lstset{columns=fixed,basicstyle=\ttfamily,numbers=left,numberstyle=\tiny,stepnumber=5,breaklines=true}
\usepackage{times}
% \usepackage[round]{natbib}
% \bibliographystyle{plainnat}
\bibliographystyle{plain}
\oddsidemargin=0cm
\evensidemargin=0cm
\newcommand{\be}{\begin{enumerate}}
\newcommand{\ee}{\end{enumerate}}
\newcommand{\bi}{\begin{itemize}}
\newcommand{\ei}{\end{itemize}}
\newcommand{\I}{\item}
\newcommand{\ty}{\texttt}
\newcommand{\kr}{K_{\rm r}}
\newcommand{\version}{0.2}
\textwidth=16cm
\textheight=23cm
\begin{document}
\title{\ty{hotspot}, v. \version: Software to Support Sperm-Typing for
Detecting Recombination Hotspots}
\author{Bernhard Haubold\\\small Max-Planck-Institute for Evolutionary Biology, Pl\"on, Germany}
\maketitle
\section{Introduction}
\ty{hotspot} is a software package for detecting and analyzing
recombination hotspots. This is carried out in three steps:
Preparation of allele-specific PCR, the actual PCR, and analysis of
the PCR products. \ty{hotspot} contains the two programs \ty{asp} and \ty{aso} for designing
allele-specific primers and oligos used in the pre-PCR phase. In
addition, it contains \ty{rr} for computing the rate of recombination
during the post-PCR phase of a project. In addition, the package
contains example input data and a script for downloading the mouse
genome sequence and the corresponding SNP data.

\section{Dependencies}
\bi
\I \ty{libdivsufsort}: \ty{https://code.google.com/p/libdivsufsort/}
\I \ty{tabix}: part of SAMtools~\cite{li09:seq},
\ty{http://www.htslib.org/}
\I \ty{gsl}: Gnu Scientific Library, \ty{http://www.gnu.org/software/gsl/}
\ei

\subsection{\ty{aso}}
\ty{aso} is a program for designing allele-specific oligonucleotides that are
complementary to SNPs contained in recombination hotspots. The program
can also find oligos that do not intersect any SNPs or indels. 

\subsection{\ty{asp}}
\ty{asp} is a program for designing PCR primers that have a 3'-end
complementary to SNPs in the regions flanking a recombination
hotspot. The user can set a maximal and a minimal primer
length. Within these bounds the program searches for the primer length
that comes closest to an optimal GC-content, which the user can also
set.

\subsection{\ty{rr}}
\ty{rr} implements a maximum likelihood procedure for estimating the
number of crossover events from the observed number of positive and
negative allele-specific PCR reactions. In addition, it uses the
likelihood ratio method for estimating a confidence interval. By
default this interval is set to 95\%, but the user can set it to any
level desired.

\section{Getting Started}
\ty{aso} and \ty{asp} are written in C on a computer running Linux and
they should
work on any standard UNIX system. However, please contact me at
\ty{haubold@evolbio.mpg.de} if you have any problems with the
programs. The following sections give a tutorial introduction to using
the \ty{aso} and \ty{asp}.

\subsection{\ty{aso}}
\bi
\I List options
\begin{verbatim}
ls aso -h
\end{verbatim}
\I Hotspot locations are supplied in a tab-delimited file containing
three columns: chromosome, start, and end. Take a look at two the
example hotspot coordinates supplied with the program
\begin{verbatim}
cat data/mus/exampleHotSpots19.txt 
# NB: These mouse hotspot data by Smagulova et al. (2011) are provided 
#   as part of the aspro software package for designing
#   Allele-Specific PRimers and Oligos. The hotspot coordinates by
#   Smagulova et al. refer to mm9 and will therefore not match the
#   current mouse assembly.
# Rerefence: Smagulova et al. (2011). Genome-wide
#   analysis reveals novel molecular features of mouse recombination
#   hotspots. Nature, 472:375-378.
# int   chr	start	end
int1	chr19	3796569	3799969
int2	chr19	3804782	3808182
\end{verbatim}
Rows starting with a hash are comments and there can be as many
comments as you like. This is followed by the hotspot data in four
columns: Hotspot name, chromosome, start, and end.
\I The two example hotspots come from chromosome 19. To load the
genome and SNP data for chromosome 19, execute 
\begin{verbatim}
make get-chr19-data-mus 
\end{verbatim}
\I Check that the genome data was downloaded:
\begin{verbatim}
ls data/mus/genome/
\end{verbatim}
\I Check that the SNP data was downloaded:
\begin{verbatim}
ls data/mus/vcf/
\end{verbatim}
\I Now run \ty{aso}
\begin{verbatim}
aso -g data/mus/genome -s data/mus/vcf data/mus/exampleHotSpots19.txt |
head
int1	rs37745172	19	3796745	GGCATAAGCCTGTGGTT	GGCATAAGTCTGTGGTT
int1	rs214526021	19	3797057	GTGAGTTCGAGGCCAGC	GTGAGTTCAAGGCCAGC
int1	rs232006218	19	3797126	AGCTTTTCCTCTTTTCT	AGCTTTTCTTCTTTTCT
int1	rs254266733	19	3797132	TCCTCTTTTCTGCCTTT	TCCTCTTTCCTGCCTTT
int1	rs212133110	19	3797168	CAGCCTCCTTGTTACTC	CAGCCTCCCTGTTACTC
int1	rs243001379	19	3797209	TAGTAGCAGTTCGGCTC	TAGTAGCAATTCGGCTC
int1	rs37665146	19	3797213	AGCAGTTCGGCTCATAC	AGCAGTTCAGCTCATAC
int1	rs221355178	19	3797237	CATTGTTGCTGTTGCCA	CATTGTTGTTGTTGCCA
int1	rs232704341	19	3797253	ACAGTGGTTTCTTGTGT	ACAGTGGTCTCTTGTGT
int1	rs250384293	19	3797258	GGTTTCTTGTGTTTGGA	GGTTTCTTATGTTTGGA
\end{verbatim}
The output consists of four columns:
\be
\I Interval name
\I SNP name
\I Chromosome
\I SNP position
\I First oligo
\I Second oligo
\ee
The two oligos only differ in the middle, the SNP position.
\I Instead of printing allele-specific oligos, \ty{aso} can also find
oligos that do not span any polymorphism. These are usually longer,
say 100 bp (\ty{-l}) and are found in the non-SNP mode (\ty{-n}):
\begin{verbatim}
aso -n -l 100 -g data/mus/genome -s data/mus/vcf data/mus/exampleHotSpots19.txt | head
int1	rs37745172	19	3796745	0.893	CCT...
int1	rs251742220	19	3796933	1.000	GCA...
int1	rs214526021	19	3797057	0.825	GAG...
int1	rs212133110	19	3797168	0.928	TTG...
int1	rs219156104	19	3797314	0.928	GGC...
int1	rs225529556	19	3797430	0.825	CAA...
int1	rs37178516	19	3797568	0.962	TAG...
int1	rs250108111	19	3797711	0.962	GTT...
int1	rs219305977	19	3797816	0.962	TGG...
int1	rs255329313	19	3797981	0.928	CTA...
\end{verbatim}
The columns indicate
\begin{enumerate}
\item Interval name
\item Name of SNP upsream of which the oligo is located
\item Start position of oligo
\item Complexity of oligo. This measure lies between zero and 1 with
  zero being least complex; a sequence consisting of a single
  nucleotide would have zero complexity; random sequences with
  equi-probable nucleotides have a complexity close to 1. We can sort
  the output according to complexity:
\begin{verbatim}
aso -n -l 100 -g data/mus/genome -s data/mus/vcf data/mus/exampleHotSpots19.txt | 
sort -n -k 5 | head
int1	rs587144268	19	3798964	0.447	ACG...
int2	rs214880222	19	3807503	0.515	CCC...
int1	rs262734534	19	3799720	0.687	TTT...
int1	rs214526021	19	3797057	0.825	GAG...
int1	rs225529556	19	3797430	0.825	CAA...
int1	rs246740259	19	3799189	0.825	ATA...
int2	rs252557020	19	3807609	0.825	TTT...
int2	rs223077599	19	3806147	0.859	TCT...
int2	rs234318596	19	3806539	0.859	ACT...
int2	rs255109107	19	3806426	0.859	TAC...
\end{verbatim}
Notice the long microsatellelite in the top sequence. 
\item Sequence
\end{enumerate}
\I Run \ty{asp} to find forward primers in a window of 5kb downstream
of the start of the interval of interest:
\begin{verbatim}
asp -g data/mus/genome -s data/mus/vcf data/mus/exampleHotSpots19.txt | head
int1	rs36588234	19	3791591	GATTCAGCCAACAA	0.29	GATTCAGCCAACAG	0.33
int1	rs266164282	19	3791605	TTTAGATAGTTGGT	0.19	TTTAGATAGTTGGG	0.24
int1	rs233070165	19	3791753	AGAAGAGCAGTCGG	0.38	AGAAGAGCAGTCGA	0.33
int1	rs586881213	19	3791766	GGTGCTCTTACCCA	0.38	GGTGCTCTTACCCT	0.38
int1	rs258256279	19	3791984	AGTTCCAGGACAGT	0.33	AGTTCCAGGACAGC	0.38
int1	rs215047461	19	3792005	GCACATAAAAACTT	0.19	GCACATAAAAACTC	0.24
int1	rs38604711	19	3792235	TATTTTGATTGCAGAAATT	0.15	TGATTGCAGAAATC	0.24
int1	rs212603417	19	3792310	TTAAACCTCCCATT	0.24	TTAAACCTCCCATC	0.29
int1	rs36316803	19	3792353	TATTGTCATGAGCA	0.24	TATTGTCATGAGCG	0.29
int1	rs579758610	19	3792405	CATTACCCATGAGC	0.33	CATTACCCATGAGG	0.33
\end{verbatim}
The output columns are
\begin{enumerate}
\I Interval name
\I SNP name
\I Chromosome
\I SNP position
\I Primer for first allele
\I GC-content for first primer
\I Primer for second allele
\I CG-content for second primer
\end{enumerate}
Notice that the two primers need not be of the same length. By
default, \ty{asp} searches for a primer of length 14--19 with a
GC-content as close to 0.5 as possible.
\I To get the reverse primers in the upstream region, run
\begin{verbatim}
asp -r -g data/mus/genome -s data/mus/vcf data/mus/exampleHotSpots19.txt | head
int1	rs249017060	19	3800119	AGAGTGAGTTCCAG	0.50	AGAGTGAGTTCCAA	0.43
int1	rs218978784	19	3800120	CAGAGTGAGTTCCA	0.50	CAGAGTGAGTTCCC	0.57
int1	rs235336981	19	3800166	GCATCCTTTAATCA	0.36	GCATCCTTTAATCC	0.43
int1	rs584116440	19	3800176	CAGTGGTGATGCAT	0.50	CAGTGGTGATGCAC	0.57
int1	rs253442444	19	3800434	CTCTCAAGTGCTGG	0.57	CTCTCAAGTGCTGA	0.50
int1	rs214858261	19	3800484	TGCTCTATGAACCA	0.43	TGCTCTATGAACCT	0.43
int1	rs232338632	19	3800485	TTGCTCTATGAACC	0.43	TTGCTCTATGAACT	0.36
int1	rs261835571	19	3800506	TGGCTGTGGCTGTC	0.64	TGGCTGTGGCTGTT	0.57
int1	rs222895727	19	3800511	TTATGTGGCTGTGG	0.50	TTATGTGGCTGTGT	0.43
int1	rs579285301	19	3800535	TATTTATTTATTTTTGAGG	0.16	ATTTATTTATTTTTGAGA	0.11
\end{verbatim}
\ei

\bibliography{/home/haubold/References/references}
\end{document}

